% (c) 2002 Matthew Boedicker <mboedick@mboedick.org> (original author) http://mboedick.org
% (c) 2003-2007 David J. Grant <davidgrant-at-gmail.com> http://www.davidgrant.ca
% (c) 2008 Nathaniel Johnston <nathaniel@nathanieljohnston.com> http://www.nathanieljohnston.com
% (c) 2011 Scott Clark <sc932@cornell.edu> http://cam.cornell.edu/~sc932
% (c) 2013 Yi Chao <yichao@mail.bnu.edu.cn> http://blog.51isoft.com
% (c) 2013 Qingwen GUAN <buaa.gg@gmail.com>
%
%This work is licensed under the Creative Commons Attribution-Noncommercial-Share Alike 2.5 License. To view a copy of this license, visit http://creativecommons.org/licenses/by-nc-sa/2.5/ or send a letter to Creative Commons, 543 Howard Street, 5th Floor, San Francisco, California, 94105, USA.
\documentclass[letterpaper,11pt,UTF8,nofonts]{ctexart}
\newlength{\outerbordwidth}
\pagestyle{empty}
\raggedbottom
\raggedright
\usepackage[svgnames]{xcolor}
\usepackage{framed}
\usepackage{tocloft}
\setCJKmainfont[BoldFont={Heiti SC},ItalicFont={STKaiti}]{STSong}
%-----------------------------------------------------------
%Edit these values as you see fit
\setlength{\outerbordwidth}{3pt}  % Width of border outside of title bars
\definecolor{shadecolor}{gray}{0.75}  % Outer background color of title bars (0 = black, 1 = white)
\definecolor{shadecolorB}{gray}{0.93}  % Inner background color of title bars

%-----------------------------------------------------------
%Margin setup
\setlength{\evensidemargin}{-0.25in}
\setlength{\headheight}{-0.25in}
\setlength{\headsep}{0in}
\setlength{\oddsidemargin}{-0.25in}
\setlength{\paperheight}{11in}
\setlength{\paperwidth}{8.5in}
\setlength{\tabcolsep}{0in}
\setlength{\textheight}{9.75in}
\setlength{\textwidth}{7in}
\setlength{\topmargin}{-0.3in}
\setlength{\topskip}{0in}
\setlength{\voffset}{0.1in}
%-----------------------------------------------------------
%Custom commands
\newcommand{\resitem}[1]{\item #1 \vspace{-2pt}}

%\newcommand{\resheading}[1]{\vspace{8pt} \textbf{\Large #1 \vphantom{p\^{E}}}\vspace{-5pt}}
\newcommand{\resheading}[1]{\vspace{5pt}

    \parbox{\textwidth}{\setlength{\FrameSep}{\outerbordwidth}
    \begin{shaded}
        \setlength{\fboxsep}{0pt}
        \framebox[\textwidth][l]{\setlength{\fboxsep}{4pt}\fcolorbox{shadecolorB}{shadecolorB}{\textbf{\sffamily{\mbox{~}\makebox[6.762in][l]{\large #1} \vphantom{p\^{E}}}}}}
    \end{shaded}
  }\vspace{-8pt}
}

\newcommand{\ressubheading}[4]{
    \begin{tabular*}{6.5in}{l@{\cftdotfill{\cftsecdotsep}\extracolsep{\fill}}r}
        \textbf{#1} & #2 \\
        \textit{#3} & \textit{#4} \\
    \end{tabular*}\vspace{-6pt}
}


\newcommand{\ressubheadingsingleline}[2]{
    \begin{tabular*}{6.5in}{l@{\cftdotfill{\cftsecdotsep}\extracolsep{\fill}}r}
        \textbf{#1} & #2 \\
    \end{tabular*}\vspace{-6pt}
}
%-----------------------------------------------------------
\begin{document}
    \begin{tabular*}{7in}{l@{\extracolsep{\fill}}r}
        \textbf{\Large 管 清文} & \textbf{\today} \\
        buaa.gg@gmail.com & \\
	(86) 134 0115 7995 &  \\
	https://github.com/buaagg & \\
    \end{tabular*}
    %%%%%%%%%%%%%%%%%%%%%%%%%%%%%%
    \resheading{教育经历}
    %%%%%%%%%%%%%%%%%%%%%%%%%%%%%%
    \begin{itemize}
    \item
        \ressubheadingsingleline{北京航空航天大学~软件工程专业~硕士}{2012 - 2015}
    \item
        \ressubheadingsingleline{北京航空航天大学~软件工程专业/应用数学专业~学士}{2008 - 2012}
    \end{itemize}

	\resheading{所获荣誉}
	%%%%%%%%%%%%%%%%%%%%%%%%%%%%%%
	\begin{itemize}
		\item
		    \begin{tabular*}{6.5in}{l@{\cftdotfill{\cftsecdotsep}\extracolsep{\fill}}r}
			\textbf{[世界级]ACM-ICPC国际大学生程序设计竞赛世界总决赛} & 第27名 \\
			\textbf{[亚洲级]ACM-ICPC国际大学生程序设计竞赛亚洲区域赛} & 第5名 \\
			\textbf{[全国级]ACM-ICPC中国东北地区大学生程序设计竞赛暨中俄大学生对抗赛} & 第2名 \\
			\textit{和2名队友配合,用C++/Java在5个小时内设计算法并编程求解指定的问题} & \textit{2012-2013}
		    \end{tabular*}\vspace{-6pt}
%		\item \ressubheading{[全国级]编程之美全国挑战赛}{全国30强}{由微软亚太研发组织举办,决赛为结对编程:Bing搜索的大数据日志分析和数据呈现}{2013}
%		\item \ressubheading{[全国级]完美未来之星智能机开发全国挑战赛}{第4名}{由完美世界举办,决赛为3-4人组队,进行智能机应用开发}{2013}
		\item \ressubheading{[全国级]渣打编程马拉松}{前3名}{由渣打银行举办,决赛为3人组队,LOG数据的挖掘与分析}{2013}
		\item \ressubheading{[全国级]西山居程序设计全国挑战赛}{第5名}{由金山西山居举办,决赛为个人赛,使用C/C++设计算法并编写程序求解指定的问题}{2013}
%		\item \ressubheadingsingleline{[校级]三国杀3v3比赛团体季军}{2013}
	\end{itemize}

    %%%%%%%%%%%%%%%%%%%%%%%%%%%%%%
    \resheading{技能}
    %%%%%%%%%%%%%%%%%%%%%%%%%%%%%%
    \begin{itemize}
        \item {\bf 开发语言:} C/C++, Java, C\#, Python, PHP.
        \item {\bf 兴趣方向:} 算法、数学。
%        \item 较强的学习能力,对新技术有较高的热情,勇于尝试新鲜事物。
    \end{itemize}

    %%%%%%%%%%%%%%%%%%%%%%%%%%%%%%
    \resheading{社会活动}
    %%%%%%%%%%%%%%%%%%%%%%%%%%%%%%
    \begin{itemize}
        \item \ressubheadingsingleline{北京航空航天大学ACM/ICPC队学生教练}{2012-2013}
	\begin{itemize}
		\item 负责维护ACM校队服务器和实验室网站的搭建。
		\item 负责校内选拔赛的组织筹备和题目准备。
	\end{itemize}
    \end{itemize}


    %%%%%%%%%%%%%%%%%%%%%%%%%%%%%%
    \resheading{项目经历}
    %%%%%%%%%%%%%%%%%%%%%%%%%%%%%%
    \begin{itemize}
    	\item \ressubheadingsingleline{基于Kinect的演示辅助系统}{2011-2012}
	\begin{itemize}
		\item {使得展示更加生动。让展示者可以通过手势和动作更加轻松的控制展示元素。最重要的是,可以让展示者可以在不借助任何外界仪器的情况下直接在屏幕上书写和圈画。}
		\item {我的工作是算法的设计与实现。大约10,000行C\#代码。}
		\item 在大学生科研训练计划(SRTP)中成绩为“优秀”。
	\end{itemize}
    \end{itemize}

    %%%%%%%%%%%%%%%%%%%%%%%%%%%%%%
   % \resheading{实习经历}
    %%%%%%%%%%%%%%%%%%%%%%%%%%%%%%
    %\begin{itemize}
   % 	\item \ressubheadingsingleline{微软亚洲研究院}{2013}
%	\begin{itemize}
%		\item 机器学习,预测用户点击广告的概率。
%	\end{itemize}
%    \end{itemize}

    %%%%%%%%%%%%%%%%%%%%%%%%%%%%%%
 %   \resheading{个人信息}
    %%%%%%%%%%%%%%%%%%%%%%%%%%%%%%
  %  \begin{itemize}
  %      \item {\bf 兴趣爱好:} 三国杀、看视频
  %  \end{itemize}

\end{document}
